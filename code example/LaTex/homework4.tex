\documentclass[12pt]{article}

\title{CS177 Homework 4: Continuous Variables \& Conditionings}
\author{Yating Liu 10588498}
\date{2018/11/7}

\setlength{\topmargin}{0.0in}
\setlength{\headheight}{0.0in}
\setlength{\headsep}{0.0in}
\setlength{\textheight}{9.0in}
\setlength{\oddsidemargin}{0in}
\setlength{\textwidth}{6.5in}

\usepackage{graphicx}
\usepackage{enumerate}
\usepackage{amsmath}
\usepackage{amssymb}
     
\begin{document}

\maketitle

\noindent

\subsubsection*{Question 1:} 
\begin{enumerate}[a)]
  \item 
  $P(X>18) = 1-P(X \leq 18) = 1 - F(X=18) = 0.0062$

  \item 
  $P(10 \leq X \leq 16) = F(X=16) - F(X=10) = 0.866$

  \item
  \begin{align*}
  \int_{-\infty}^{13} xf(x) &= 0.99 \\
  \int_{-\infty}^{\frac{13-\mu}{\sigma}} zf(z) &= 0.99 \\
  \end{align*}
  using \emph{norm.ppf}, we get $\dfrac{13-\mu}{\sigma}=2.326$, therefore, the average runtime should be 8.348.
  

 
\end{enumerate}

\subsubsection*{Question 2:} 
\begin{enumerate}[a)]
  \item 
  we would like to predict $Y=1$ only when
  \begin{equation*}
  \dfrac{f_{x \mid y}(X \mid 1)}{f_{x \mid y}(X \mid 0)} \geq \dfrac{P(Y=0)}{P(Y=1)}
  \end{equation*}
  because $P(Y=0)=P(Y=1)$, we then have
  \begin{equation*}
  f_{x \mid y}(X \mid 1) \geq f_{x \mid y}(X \mid 0)
  \end{equation*}
  assume when $Y=0$, mean is $\mu_{0}$, when $Y=1$, mean is $\mu_{1}$
  \begin{align*}
  \theta_{1}e^{-\theta_{1}x} &\geq \theta_{0}e^{-\theta_{0}x}\\
  \frac{1}{50}e^{-\frac{1}{50}x} &\geq e^{-x}\\
  \frac{e^{-x}}{e^{-\frac{1}{50}x}} &\leq \frac{1}{50}\\
  \ln{(e^{-x})}-\ln{(e^{-\frac{1}{50}x)}} &\leq \ln{\left(\frac{1}{50}\right)}\\
  -\dfrac{49}{50}x &\leq -3.912\\
  x &\geq 3.992\\
  \end{align*}
  % make sure no empty line when using aligh  
  Therefore, when $x \geq 3.992$, we would like to predict $Y=1$, and when $x<3.992$ we would like to predict $y=0$.
  
  \item 
  Using the same formula in part(a) with $P(Y=0)=0.99$ and $P(Y=1)=0.01$.
  \begin{align*}
  \dfrac{f_{x \mid y}(X \mid 1)}{f_{x \mid y}(X \mid 0)} &\geq \dfrac{P(Y=0)}{P(Y=1)}\\
  \dfrac{f_{x \mid y}(X \mid 1)}{f_{x \mid y}(X \mid 0)} &\geq \dfrac{0.99}{0.01}\\
  \theta_{1}e^{-\theta_{1}x} &\geq 99\theta_{0}e^{-\theta_{0}x}\\
  \frac{1}{50}e^{-\frac{1}{50}x} &\geq 99e^{-x}\\
  \frac{e^{-x}}{e^{-\frac{1}{50}x}} &\leq \frac{1}{50 \cdot 99}\\
  \ln{(e^{-x})}-\ln{(e^{-\frac{1}{50}x)}} &\leq \ln{\left(\frac{1}{50 \cdot 99}\right)}\\
  -\dfrac{49}{50}x &\leq -8.507\\
  x &\geq 8.6806\\
  \end{align*}
  Therefore, when $x \geq 8.6806$, we would like to predict $Y=1$, and when $x<8.6806$ we would like to predict $y=0$.
  
  \item 
  Using the formula in lecture slide, we have
  \begin{align*}
  \dfrac{f_{x \mid y}(X \mid 1)}{f_{x \mid y}(X \mid 0)} &\geq \dfrac{P(Y=0)}{P(Y=1)} \cdot \left(\dfrac{\lambda_{01}}{\lambda_{10}}\right)\\
  \dfrac{f_{x \mid y}(X \mid 1)}{f_{x \mid y}(X \mid 0)} &\geq \dfrac{0.99}{0.01} \cdot \dfrac{1}{500}\\
  \theta_{1}e^{-\theta_{1}x} &\geq \dfrac{99}{500}\theta_{0}e^{-\theta_{0}x}\\
  \frac{1}{50}e^{-\frac{1}{50}x} &\geq \dfrac{99}{500}e^{-x}\\
  \frac{e^{-x}}{e^{-\frac{1}{50}x}} &\leq 0.101\\
  \ln{(e^{-x})}-\ln{(e^{-\frac{1}{50}x)}} &\leq \ln{(0.101)}\\
  -\dfrac{49}{50}x &\leq -2.293\\
  x &\geq 2.340\\
  \end{align*}  Therefore, when $x \geq 2.340$, we would like to predict $Y=1$, and when $x<2.340$ we would like to predict $y=0$.



\end{enumerate}


\subsubsection*{Question 3:} 
\begin{enumerate}[a)]
  \item 
  The expected total waiting time is $\dfrac{1}{\alpha}+\dfrac{1}{\beta}$
  
  \item 
  The variance of the waiting time of the first queue is $\dfrac{1}{\alpha^2}$, of the second queue is $\dfrac{1}{\beta^2}$. Because the waiting time is independent between two queues, the standard deviation of the total waiting time is $\sqrt{\dfrac{1}{\alpha^2}+\dfrac{1}{\beta^2}}$.
  
     
  \item 
  Assume we use \emph{t} to denote the time taken by the first queue, \emph{s} to denote the total time
  \begin{equation*}
  f(s) = \int_{0}^{s}\alpha e^{-\alpha t}\beta e^{-\beta (s-t)} = \int_{0}^{s}\alpha \beta e^{-\alpha t -\beta s + \beta t} = \alpha \beta e^{-\beta s}\int_{0}^{s}e^{(\beta -\alpha) t} = \dfrac{\alpha \beta}{\beta -\alpha}\left(e^{-\alpha s}-e^{-\beta s}\right)
  \end{equation*}
  The plot I drawn is in the last part. 


\end{enumerate}

\subsubsection*{Question 4:} 
\begin{enumerate}[a)]
  \item 
  \begin{equation*}
  \ln{f{_{x\mid y}(x_i\mid y_i=1)}} = \sum\limits_{j=1}^{M}-\dfrac{1}{2}ln{(2\pi \sigma_{1j}^2)} - \dfrac{(x_{ij}-\mu_{1j})^2}{2\sigma_{1j}^2}
  \end{equation*}
  \begin{equation*}
  \ln{f{_{x\mid y}(x_i\mid y_i=0)}} = \sum\limits_{j=1}^{M}-\dfrac{1}{2}ln{(2\pi \sigma_{0j}^2)} - \dfrac{(x_{ij}-\mu_{0j})^2}{2\sigma_{0j}^2}
  \end{equation*}
     
  \item 
  Because $p_Y(0)=p_Y(1)=\frac{1}{2}$, we will predict $Y_i=1$ only when
  \begin{equation*}
  \ln{f{_{x\mid y}(x_i\mid y_i=1)}} > \ln{f{_{x\mid y}(x_i\mid y_i=0)}}
  \end{equation*}
  The classification accuracy is 0.5833, the number of false alarms is 267, the number of missed detections is 3. 
  \item 
  The classification accuracy is 0.7392, the number of false alarms is 167, the number of missed detections is 2. 
  
  \item 
  The classification accuracy is 0.75, the number of false alarms is 158, the number of missed detections is 4. 
  



\end{enumerate}
  
  

\end{document}